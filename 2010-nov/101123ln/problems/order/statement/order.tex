% Author: Ivan Kazmenko
% Text author: Ivan Kazmenko
% Origin: 20081013 - SPbSU First Year Training 5 - Graph Theory 2
\begin{problem}{Порядок циклов}
{order.in}{order.out}
{2 секунды}{256 мебибайт}{}

Дан неориентированный граф из $n$ вершин, заданный матрицей смежности
\t{a} (\t{a[u][u] = True}; \t{a[u][v] = a[v][u]}; \t{a[u][v] = True}
тогда и только тогда, когда есть ребро между вершинами \t{u} и \t{v}).
На нём запускают следующий алгоритм:
\begin{verbatim}
for x := 1 to n do
  for y := 1 to n do
    for z := 1 to n do
      if a[i][k] and a[k][j] then
        a[i][j] := True;
\end{verbatim}
Перед запуском буквы \t{x}, \t{y} и \t{z} заменяют буквами
\t{i}, \t{j} и \t{k} в некотором порядке. Утверждается, что после работы
этого алгоритма \t{a[u][v] = True} тогда и только тогда, когда в исходном
графе существует путь между вершинами \t{u} и \t{v}. Выясните, верно ли это,
и если нет, приведите пример исходного графа, на котором это неверно.

\InputFile

В первой строке входного файла записаны через пробел три буквы "---
`\t{i}', `\t{j}' и `\t{k}' "--- в некотором порядке. Первая буква
подставляется в программу вместо `\t{x}', вторая "--- вместо `\t{y}',
третья "--- вместо `\t{z}'.

\OutputFile

Если искомый граф существует, в первой строке выходного файла выведите
через пробел целые числа $n$ и $m$ "--- количество вершин и рёбер в графе,
соответственно ($1 \le n \le 10$, $0 \le m \le 45$).
В следующих $m$ строках выведите пары вершин, соединённых рёбрами, по одной
паре на строке. Номера вершин в паре должны быть упорядочены по возрастанию;
вершины нумеруются с единицы. Кратные рёбра и петли не допускаются.

Если же программа с заданным порядком циклов корректно работает на любом
графе, вместо $n$ и $m$ выведите в первой строке два нуля через пробел.

\Example

\begin{example}
\exmp{
k i j
}{
0 0
}%
\end{example}

\end{problem}
